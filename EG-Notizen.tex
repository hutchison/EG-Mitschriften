\documentclass[%
a4paper, %apaper, % alle weiteren Papierformat einstellbar
%empty,           % keine Seitenzahlen
9pt,              % Schriftgröße (12pt, 11pt (Standard))
leqno,            % Nummerierung von Gleichungen links
fleqn,            % Ausgabe von Gleichungen linksbündig
]
{scrartcl}

%\pagestyle{headings}

%% Deutsche Anpassungen
\usepackage[ngerman]{babel}
\usepackage[T1]{fontenc}

\usepackage[sc]{mathpazo}
\linespread{1.05}

\usepackage[utf8]{inputenc}

%obligatorischer Mathekram:
\usepackage{amssymb,amstext,amsthm}
\usepackage{stmaryrd}
\usepackage[sumlimits]{amsmath}
\usepackage{eulervm}

%nützliche Extras:
\usepackage{array,
  %hhline,
  %longtable,
  %tabularx,
  %enumerate,
  %hyperref,
  %color,
  %setspace,
  %booktabs,
  %cite,
  %caption,
  %lineno,
  %lastpage,
  %algorithm,
  %algorithmic,
  %listings,
  %bussproofs,
  ulem,
  subfig,
}

%nötig für Querformat
%\usepackage[landscape]{geometry}

%Kommando für Schriftarten
\newcommand{\changefont}[3]{\fontfamily{#1}\fontseries{#2}\fontshape{#3}\selectfont}

%bietet gestrichelte vert. Linien in Tabellen (':')
%\usepackage{arydshln}

%nützlicher Mathekram:
\newcommand{\R}{\mathbb{R}}
\newcommand{\C}{\mathbb{C}}
\newcommand{\Z}{\mathbb{Z}}
\newcommand{\K}{\mathbb{K}}
\newcommand{\N}{\mathbb{N}}
\newcommand{\Q}{\mathbb{Q}}
\newcommand{\F}{\mathbb{F}}
\newcommand{\E}{\mathbb{E}}
\renewcommand{\P}{\mathbb{P}}
\newcommand{\off}{\text{off}\,}
\newcommand{\rg}{\text{rg}\,}
\newcommand{\Sp}{\text{Sp}\,}
\newcommand{\charpol}{\text{charpol}\,}
\newcommand{\id}{\text{id}\,}
\newcommand{\kgV}{\text{kgV}}
\newcommand{\ggT}{\text{ggT}}
\newcommand{\ora}{\overrightarrow}
\newcommand{\TV}{\text{TV}}
\newcommand{\Aut}{\text{Aut}}
\newcommand{\Zw}{\textit{Zw}}
\newcommand{\Rot}[2]{\text{Rot}_{#1,#2}}
\newcommand{\conv}{\text{conv}}

%\usepackage[left=1cm,right=1cm,top=1cm,bottom=1cm,includeheadfoot]{geometry}
\usepackage[
  %cm,
  headings
]{fullpage}

\usepackage{fancyhdr}
\pagestyle{fancy}
\fancyhf{}

\fancyhead[L]{Effiziente Graphenalgorithmen}
\fancyhead[C]{Notizen zur Vorlesung}
\fancyhead[R]{4. April 2012}

% \fancyfoot[L]{}
\fancyfoot[C]{\thepage}
% \fancyfoot[R]{\thepage / \pageref{LastPage}}

%Linie oben/unten
%\renewcommand{\headrulewidth}{0.0pt}
\renewcommand{\footrulewidth}{0.0pt}

\parindent 0pt

%% Packages für Grafiken & Abbildungen
\usepackage{pgf,tikz}      %%TeX ist kein Zeichenprogramm
\usetikzlibrary{arrows,calc,through}

%\usepackage{eso-pic}
%\AddToShipoutPicture*{
  %\put(\LenToUnit{0mm},\LenToUnit{228.5mm}){\rule{\LenToUnit{20pt}}{\LenToUnit{0.5pt}}}
  %\put(\LenToUnit{0mm},\LenToUnit{137mm}){\rule{\LenToUnit{20pt}}{\LenToUnit{0.5pt}}}
  %\put(\LenToUnit{0mm},\LenToUnit{68.5mm}){\rule{\LenToUnit{20pt}}{\LenToUnit{0.5pt}}}
%}

\begin{document}

\section*{Grundbegriffe} % (fold)
\label{sec:Grundbegriffe}

\begin{description}
  \item[Graph:] Ein Graph $G = (V,E)$ ist ein geordnetes Paar, das aus den
    Mengen der Knoten $V$ (für \textit{vertices}) und Kanten $E$ (für
    \textit{edges}) besteht.
  \item[Knotenmenge:] ist eine stinknormale Menge.
  \item[Kantenmenge:] ist eine Menge, die aus Zweiermengen besteht, deren
    Elemente aus $V$ sind. Notation: $e = \{u,v\} \in E$, verkürzend auch $e =
    uv$
  \item[verbunden, benachbart, adjazent:] Zwei Knoten $u,v$ sind verbunden
    (benachbart, adjazent), wenn eine Kante $e$ mit $e = uv$ existiert.
  \item[inzident] Wir sagen: die zwei Knoten einer Kante inzidieren mit ihr.
    Soll heißen: sie liegen auf bzw. an der Kante.
  \item[Nachbarschaft:] die Menge von Knoten, die mit einem bestimmten Knoten
    verbunden sind. Genauer: $N(v) = \{ u : vu \in E \}$
  \item[endlicher Gr.:] ein Gr. der nur endlich viele Knoten und Kanten hat
  \item[Nullgraph] ein Graph ohne Knoten und Kanten
  \item[trivialer Gr.:] ein Gr. mit nur einem Knoten. Alle anderen nennt man
    nichttriviale Graphen.
  \item[Schleife:] eine Kante, die einen Knoten mit sich selbst verbindet
  \item[parallele Kanten:] mehrere Kanten mit identischen Endknoten
  \item[einfacher Gr.] ein Gr. ohne Schleifen und parallele Kanten
  \item[Grad eines Knotens:] $\deg(v)=$ Anzahl der Kanten an $v$ ($=|N(v)|$)
  \begin{description}
    \item[Maximalgrad:] $\Delta(G)$
    \item[Minimalgrad:] $\delta(G)$
    \item[regulärer Gr.] alle Knoten haben den gleichen Grad
  \end{description}
  \item[unabhängige Knotenmenge:] eine Teilmenge der Knoten, die nicht
    miteinander verbunden ist. Genauer: $U \subset V: uv \notin E$ für alle $u,v
    \in U$
  \item[Unabhängigkeitszahl:] maximale Anzahl unabhängiger Knoten. Genauer:
    $\alpha(G) = \max\{|U| : U \text{ ist unabhängig in } G\}$
  \item[Clique:] eine Teilmenge von $V$ in der alle Knoten miteinander verbunden
    sind. Genauer: $Q \subset V: uv \in E \text{ für alle } u \neq v \in Q$
  \item[Cliquenzahl:] die Größe der größten Clique, genauer $\omega(G) =
    \max\{|Q| : Q \text{ ist Clique in } G\}$
  \item[Komplementärgraph:] der Graph, bei dem alle vorher nicht verbundenen
    Knoten nun miteinander verbunden sind. Genauer: Komplementärgraph von $G =
    (V,E): \overline{G} = \left( V, \binom{V}{2} \setminus E \right)$
  \item[Inzidenzmatrix:] zeigt, welche Knoten wie oft auf welcher Kante liegen.
    Genauer: Die Inzidenzmatrix von $G$ ist die $n \times m$ Matrix
    $\mathbf{M}_G := (m_{ve})$ mit $n = |V|$, $m = |E|$ und $m_{ve} = $ Anzahl
    (0, 1 oder 2) der Inzindenzen von $v$ auf $e$.
  \item[Adjazenzmatrix:] zeigt, wie oft zwei Knoten durch eine Kante verbunden
    sind. Genauer: ist die $n \times n$ Matrix $\mathbf{A}_G := (a_{uv})$ mit
    $a_{uv}$ ist die Anzahl der Kanten zwischen $u$ und $v$, wobei eine Schleife
    als doppelte Kante zählt.
    \begin{figure}[h]
      \centering
      \subfloat[Graph]{
        \begin{minipage}[h]{0.2\textwidth}
          \begin{tikzpicture}[node distance=1.5cm]
            \node (u) [circle,draw] {$u$};
            \node (v) [circle,draw,right of=u] {$v$};
            \node (w) [circle,draw,below of=v] {$w$};
            \node (x) [circle,draw,below of=u] {$x$};
            \node (y) [circle,draw,left of=x] {$y$};
            \path (u) edge node [above] {$a$} (v);
            \path (u) edge [-,loop] node [above] {$b$} ();
            \path (v) edge node [right] {$c$} (w);
            \path (w) edge [bend left] node [below] {$d$} (x);
            \path (v) edge node [right] {$e$} (x);
            \path (w) edge [bend right] node [below] {$f$} (x);
            \path (u) edge node [left] {$g$} (x);
            \path (x) edge node [above] {$h$} (y);
          \end{tikzpicture}
        \end{minipage}
      }
      \qquad
      \subfloat[Inzidenzmatrix]{
        \begin{minipage}[h]{0.3\textwidth}
          $\begin{array}{c|*{8}{c}}
              & a & b & c & d & e & f & g & h\\\hline
            u & 1 & 2 & 0 & 0 & 0 & 0 & 1 & 0\\
            v & 1 & 0 & 1 & 0 & 1 & 0 & 0 & 0\\
            w & 0 & 0 & 1 & 1 & 0 & 1 & 0 & 0\\
            x & 0 & 0 & 0 & 1 & 1 & 1 & 1 & 1\\
            y & 0 & 0 & 0 & 0 & 0 & 0 & 0 & 1
          \end{array}$
        \end{minipage}
      }
      \quad
      \subfloat[Adjazenzmatrix]{
        \begin{minipage}[h]{0.2\textwidth}
          $\begin{array}{c|*{5}{c}}
              & u & v & w & x & y\\\hline
            u & 2 & 1 & 0 & 1 & 0\\
            v & 1 & 0 & 1 & 1 & 0\\
            w & 0 & 1 & 0 & 2 & 0\\
            x & 1 & 1 & 2 & 0 & 1\\
            y & 0 & 0 & 0 & 1 & 0
          \end{array}$
        \end{minipage}
      }
    \end{figure}
\end{description}

\subsection*{Beobachtungen} % (fold)
\label{sub:Beobachtungen}

In jedem Graphen $G = (V,E)$ gilt:
\begin{enumerate}
  \item $\sum\limits_{v \in V} \deg(v) = 2 |E|$\label{summegradknotengleich2e}
    \begin{proof}
      Man denke an die Inzidenzmatrix $\mathbf{M}$ von $G$. Die Zeilensummen
      sind genau die Grade der Knoten und damit ist $\sum\limits_{v \in V}
      \deg(v)$ die Summe aller Einträge aus $\mathbf{M}$. Die Spaltensummen sind
      immer 2, da jede Kante genau zwei Knoten verbindet (Schleifen werden
      doppelt gezählt). Somit ist die Summe der Spaltensummen (und damit die
      Summe aller Einträge) gleich $2|E|$. Da beide Summen gleich sind, gilt
      $\sum\limits_{v \in V} \deg(v) = 2 |E|$.
    \end{proof}
  \item Die Anzahl der Knoten mit ungeradem Grad ist gerade. \hfill
    (Handschlag-Lemma)
    \begin{proof}
      Man betrachte die Gleichung aus \ref{summegradknotengleich2e}. modulo 2.
      Für den Grad eines Knotens kann nur gelten:
      \begin{align*}
        \deg(v) \equiv
          \begin{cases}
            1 \mod 2, \text{falls $\deg(v)$ ungerade}\\
            0 \mod 2, \text{falls $\deg(v)$ gerade}\\
          \end{cases}
      \end{align*}
      Die rechte Seite aus \ref{summegradknotengleich2e}. ist demnach kongruent
      $0$. Die linke Seite ist die Summe der Knoten ungeraden Grads und kann nur
      kongruent $0$ sein, wenn wir es mit einem ordentlichen Graphen zu tun
      haben.
    \end{proof}
\end{enumerate}

% subsection Beobachtungen (end)

% section Grundbegriffe (end)


\section*{Teilgraphen} % (fold)
\label{sec:Teilgraphen}

\begin{description}
  \item[Teilgraph:] $H = (V', E')$ ist ein Teilgraph von $G = (V,E)$, falls $V
    \subset V$ und $E' \subset E$.

  Sei $H = (V', E')$ ein Teilgraph von $G = (V,E)$, dann definieren sich
  folgende Begriffe:

  \item[Sehne von $H$:] jede Kante $xy \in E \setminus E'$ mit $x,y \in V'$;
    also die Kanten, die im Obergraphen enthalten sind, aber nicht im Teilgraphen
  \item[$H$ ist induzierter Teilgraph:] $H$ ist sehnenlos\\
    also alle Kanten, die im Obergraphen enthalten sind, sind auch im
    Teilgraphen enthalten
  \item[$H$ ist aufspannender Teilgraph:] $V' = V$\\
    also alle Knoten, die im Obergraphen enthalten sind, sind auch im
    Teilgraphen enthalten

    \begin{figure}[ht]
      \caption{Beispiele zu Teilgraphen}
      \centering
      \subfloat[$G$]{
        \begin{tikzpicture}
          \node (v1) at (0,0)  [circle, draw] {$v_1$};
          \node (v2) at (0,1)  [circle, draw] {$v_2$};
          \node (v3) at (1,2)  [circle, draw] {$v_3$};
          \node (v4) at (2,1)  [circle, draw] {$v_4$};
          \node (v5) at (2,0)  [circle, draw] {$v_5$};
          \node (v6) at (1,-1) [circle, draw] {$v_6$};
          \path (v1) edge (v2) edge (v5) edge (v6);
          \path (v2) edge (v3) edge (v4);
          \path (v4) edge (v3) edge (v5);
          \path (v5) edge (v6);
        \end{tikzpicture}
      }
      \qquad
      \subfloat[Ein Teilgraph von $G$]{
        \begin{tikzpicture}
          \node (v1) at (0,0)  [circle, draw] {$v_1$};
          \node (v2) at (0,1)  [circle, draw] {$v_2$};
          \node (v3) at (1,2)  [circle, draw] {$v_3$};
          \node (v4) at (2,1)  [circle, draw] {$v_4$};
          \node (v5) at (2,0)  [circle, draw] {$v_5$};
          \node (v6) at (1,-1) [circle, transparent] {$v_6$};
          \path (v1) edge (v2) edge (v5);% edge (v6);
          \path (v2) edge (v3) edge (v4);
          %\path (v4) edge (v3) edge (v5);
          \path (v5) edge (v4);
        \end{tikzpicture}
      }
      \qquad
      \subfloat[Ein induzierter Teilgraph von $G$]{
        \begin{tikzpicture}
          \node (v1) at (0,0)  [circle, draw] {$v_1$};
          \node (v2) at (0,1)  [circle, draw] {$v_2$};
          \node (v3) at (1,2)  [circle, draw] {$v_3$};
          \node (v4) at (2,1)  [circle, draw] {$v_4$};
          \node (v5) at (2,0)  [circle, draw] {$v_5$};
          \node (v6) at (1,-1) [circle, transparent] {$v_6$};
          \path (v1) edge (v2) edge (v5);% edge (v6);
          \path (v2) edge (v3) edge (v4);
          \path (v4) edge (v3) edge (v5);
        \end{tikzpicture}
      }
      \qquad
      \subfloat[Ein aufspannender Teilgraph von $G$]{
        \begin{tikzpicture}
          \node (v1) at (0,0)  [circle, draw] {$v_1$};
          \node (v2) at (0,1)  [circle, draw] {$v_2$};
          \node (v3) at (1,2)  [circle, draw] {$v_3$};
          \node (v4) at (2,1)  [circle, draw] {$v_4$};
          \node (v5) at (2,0)  [circle, draw] {$v_5$};
          \node (v6) at (1,-1) [circle, draw] {$v_6$};
          \path (v1) edge (v2) edge (v6);
          \path (v2) edge (v4);
          \path (v4) edge (v3) edge (v5);
          \path (v6) edge (v5);
        \end{tikzpicture}
      }
    \end{figure}

  \item[Weg:] Ein Weg in $G$ ist eine Folge von Knoten $u_0, u_1, \dots, u_l$
    mit $u_i u_{i+1} \in E$ für alle $0 \leq i < l$
  \item[Pfad:] oder auch einfacher Weg ist ein Weg in dem kein Knoten doppelt
    vorkommt
  \item[Zykel:] ein Weg $u_0, u_1, \dots, u_l$ mit $u_0 = u_l$, also ein Weg bei
    dem Anfangs- und Endknoten gleich sind
  \item[Kreis:] oder auch einfacher Zykel ist ein Zykel mit den Knoten $u_0,
    u_1, \dots, u_l = u_0$ in dem die Knoten $u_1, \dots, u_l$ verschieden sind.
    Also alle Knoten, außer Anfangs- und Endknoten, sind verschieden.
  \item[zusammenhängender Graph:] wenn es zwischen je zwei Knoten einen Weg gibt
  \item[Komponenten:] (Zusammenhangskomponenten) die maximalen zusammenhängenden
    Teilgraphen
  \item[Artikulationspunkt] (Trennknoten) ist ein Knoten, bei dessen Wegnahme
    sich die Anzahl der Komponenten erhöht
\end{description}

\subsection*{Beobachtungen} % (fold)
\label{sub:Beobachtungen}

\begin{itemize}
  \item Es sei $G = (V,E)$ ein zusammenhängender Graph und $e$ eine Kante eines
    Kreises in $G$, dann ist $G-e$ wieder zusammenhängend.
\end{itemize}

% subsection Beobachtungen (end)

% section Teilgraphen (end)

\section*{Bäume} % (fold)
\label{sec:Bäume}

\begin{description}
  \item[Baum:] zusammenhängender kreisfreier Graph
  \item[Wald:] kreisfreier Graph; Komponenten sind demnach Bäume
  \item[Blatt:] Knoten vom Grad 1
  \item[Spannbaum:] ein aufspannender Teilgraph und Baum eines Graphen
\end{description}

\subsection*{Beobachtungen} % (fold)
\label{sub:Beobachtungen}

\begin{enumerate}
  \item Jeder Baum mit mindestens zwei Knoten hat mindestens zwei Blätter.
  \item Ist $v$ ein Blatt vom Baum $G$, dann ist $G-v$ wieder ein Baum.
  \item Ein Graph ist genau dann zusammenhängend, wenn er einen Spannbaum
    enthält.
\end{enumerate}

% subsection Beobachtungen (end)

\subsection*{Charakterisierungen von Bäumen} % (fold)
\label{sub:Charakterisierungen von Bäumen}

Für jeden Graphen $G = (V,E)$ mit $n \geq 1$ Knoten sind die folgenden
Eigenschaften äquivalent:
\begin{enumerate}
  \item $G$ ist ein Baum
  \item $G$ ist zusammenhängend und hat genau $n-1$ Kanten
  \item $G$ ist kreisfrei und hat genau $n-1$ Kanten
  \item $G$ ist kanten-maximal kreisfrei, d.\,h. $G$ ist kreisfrei und beim
    Hinzufügen einer neuen Kante entsteht ein Kreis
  \item zwischen je zwei Knoten $a,b \in V$ gibt es genau einen Weg
  \item $G$ ist Kanten-minimal zusammenhängend, d.\,h. $G$ ist zusammenhängend
    und nach Wegnahme einer Kante ist $G$ nicht mehr zusammenhängend
\end{enumerate}

% subsection Charakterisierungen von Bäumen (end)

% section Bäume (end)

\end{document}
